\section{Toteutus}

\subsection{Tehokkuuden mittaus}
Tehokkuutta mitataan tuotetun lämpöenergia ja käytettävän polttoaineen välisenä
suhteena. Suurin mahdollinen arvo on 1, jolloin tapahtuu täydellinen palaminen.

\begin{itemize}
	\item Lämmönsiirron tehokkuus
		\begin{itemize}
			\item Kuinka hyvin polttoastia siirtää energiaa ympäristöön.
			\item astian massa, kyky siirtää tai säilyttää lämpöä, lämpöhäviöt (hormi)
		\end{itemize}
	\item lämpöhäviöitä voi mitata järjestelmästä poistuvien pakokaasujen, jotka
	sisältävät energiaa polttamattoman kaasun ja partikkelimaterian muodossa, perusteella.
\end{itemize}

\subsection{Työn ulkopuolelle rajatut tekijät}

\subsection{Tekijöitä, joita optimoijan käyttäjä pystyy määrittelemään}
\begin{tabular*}{\textwidth}{lcl}
	\toprule
	\bf Nimi & \bf Suure & \bf Kuvaus \\
	\midrule
	Paloaika & \(t\) & Kuinka pitkän aikaa palamista tarkastellaan? \\
	Massa & \(m\) & Poltettavan puun massa \\
	\bottomrule
\end{tabular*}

\subsection{Geneettinen algoritmi}
Alkion genotyypin alleelit ovat liukulukuja väliltä \([0, 1]\).

