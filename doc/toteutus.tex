\section{Työn toteutus}

Ohjelmointi aloitettiin luomalla valikot ohjelmassa navigoimista varten, tätä varten luotiin apufunktioita objektiin console.scala.
Valikkoihin lisättiin pääsy mm. ongelman valintaan, muuttujien ja parametrien hallintaan sekä GAn toimintojen suorittamiseen.
Valikkojen luominen sujui ongelmitta, ja onnistuminen kokeilemalla navigointia konsoliohjelmassa. Valikkojen toiminnallisuuksiin
ei puututtu vielä tässä vaiheessa.

Tämän jälkeen alettiin luomaan luvussa \ref{komponentit} esiteltyjä komponentteja. Alkioita varten, joka on geno- ja fenotyypin yläluokka,
luotiin erillinen luokka muuttujia varten (\textit{Variable}), minkä jäseniin alkion alleelien arvot liitettiin. Muuttuja luokan oliot
sisältävät vapaasti määriteltävät nimi ja yksikkö kentät, jotka selventävät ohjelman käyttäjälle muuttujan käyttöä. Luokassa on tieto myös
kullekkin muuttuja luokan oliolle laskettavan arvon ylä- ja alarajasta, joita sovelluksen käyttäjä voi muuttaa ajon aikana.
Sovellus tukee myös vakioita, jotka voidaan ottaa käyttöön määrittelemällä ylä- ja alaraja samaksi.

Kun komponentit oltiin saatu luotua, tehtiin luokka geneettisen algoritmin hallintaa varten. Luokkaan lisättiin metodi
valvojan parametrien (mutaation todennäköisyys, risteytyksen todennäköisyys, populaation koko ja elitismi) hallintaan
sekä metodi valvojan käsittämän ongelman muuttujien hallintaan. Luokkaan tehtiin funktiot myös risteytystä ja uuden populaation
generoimista varten. Uuden populaation generoimista varten tulee valita nykyisestä populaatiosta jäsenet, jotka siirtävät tähän informaatiota.
Tätä varten luotiin oma piire (\textit{Trait, ParentSelection}), jonka jälkeen luotiin edelleen tästä periytyviä
erilaisia valintametodeja käsittäviä objekteja.

Työn aikana luotiin kolme eri valintatapaa. Jokaisessa näissä valintatavoissa valitaan riittävä määrä vanhempia seuraavan
populaation generoimsta varten. \textit{RouletteSelection}:issa alkiot järjestetään kelvollisuusarvon mukaiseen
järjestykseen, ja jonon kullakin jäsenellä on tämän kelvollisuusarvosta johdettu todennäköisyys
(\(\frac{\text{alkion kelvollisuusarvo}}{\text{Populaation yhteenlasketut kelvollisuusarvot}}\))
päätyä vanhemaksi. \textit{RankSelection:ion} ei suosi yhtä paljoa korkean kelvollisuusarvon omaavia yksilöitä,
mikä vähentää GA:n jämähtämäistä lokaaleihin maksimeihin tai minimeihin. Edellä esitelty valintamenetelmä
eroaa RouletteSelection:ista siten, että todennäköisyys alkion valinnalle määritellään sen paikasta järjestetystä
populaatiosta (\(\frac{N-n}{\Sigma n}\)), jossa \(n\) alkion indeksi, indeksien yhteenlaskettu summa
\(\Sigma n = N(N + 1)/2\) ja \(N\) populaation koko. Lisäksi luotiin \textit{RankedTournamentSelection}, joka toimii
samoin tavoin edellä mainitun menetelmän kanssa, mutta vanhemmat valitaan satunnaisesta otoksesta koko populaation
sijaan.

Tämän jälkeen luotiin luvussa \ref{suunnittelu} esitetyille ongelmille valvoja luokan oliot.
Kelvollisuusfunktioita määriteltäessä tulee ottaa huomioon, että mitä suurempi siitä palautettu arvo on,
sitä suurempi todennäköisyys alkion on päätyä seuraavaan populaatioon. Maksimointi ongelmissa kelvollisuusarvon
määrittäminen on suoraviivaista, sillä se voidaan suoraan määrittää ongelman "yhtälöksi".
Jos on tarve suorittaa minimointi ongelma, niin esimerkiksi yhtälön \ref{eq:sum} kelvollisuusarvo voitaisiin laskea seuraavasti:
\((-1)\times(a_1 - a_2 + a_3 - a_4 + a_5 - a_6 + a_7 - a_8)\). Jos taas halutaan etsiä kohtia, missä jokin yhtälö
saavuttaa tietyn arvon, voidaan esimerkiksi yhtälön \ref{eq:xyz_graph} tapauksessa funktio määrittää
\(a - \left| y \sin{ \left( \sqrt {x^2 + y^2} \right) } + x \left( sign(y) \right) \right|\), jossa \(a\) on tavoiteltava arvo.

