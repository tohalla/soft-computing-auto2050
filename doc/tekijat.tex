\section{Puun polton tehokkuuteen vaikuttavat tekijät}

Tehokkuutta mitataan systeemin syötetyn energian lämpöä tuottavan osuuden avulla.
Kaava tälle voidaan kirjoittaa auki seuraavasti.

\begin{equation}
	\eta=\frac{Q_{in}-Q_a-Q_b}{Q_{in}}
\end{equation}

\noindent
Jossa \(Q_a\) on lämpöhäviöt, ja \(Q_b\) kemialliset hävikit (kaasut, jotka häviävät ennen polttoa).
Lämpöhäviöt koostuvat vesihöyrystä ja päästöjen kuivasta osuudesta.

\begin{equation}
	Q_a=\int_0^t{
		\dot V_f c_{p,dry}(T_f-T_{0,i}) +
		\dot V_w c_{p,w}(T_f-T_{0,i})
	}\ dt
\end{equation}

\begin{equation}
	Q_b=\int_0^t{
		\dot m_{CO}LHV_{CO}
	}
\end{equation}

Jotta puu palaisi mahdollisimman täydellisesti tarvitaan riittävästi aikaa,
tarpeeksi korkea lämpötila palamisprosessin ylläpitoon sekä riittävä ilmavirtaus.
Nopeasti palava tuli on tyypillisesti `tehokkaampi` kuin hidas tuli, sillä palamisnopeuden
kasvaessa lämpötila ja ilmavirtaus kasvaa.

\subsection{Puun kosteus}
Tuore puu on usein kosteaa. Kosteuden kasvaessa puun lämpöarvo laskee,
koska osa lämmöstä kuluu veden haihduttamiseen.
Toisaalta, puun ollessa liian kuivaa, se palaa liian nopeasti,
jolloin hapen syöttö voi olla rajoittavana tekijänä maksimaalisen tehon saavuttamiseksi.

\subsection{Happi}
Palamisreaktioon saadaan happea systeemiin syötetystä ilmasta.
Polttoprosessin tehostamiseksi ilman täytyy sekoittua polttoaineeseen oikeassa lämpötilassa
ja pysyä polttoastiassa niin kauan, että palamisen kemiallinen reaktio on saatu
suoritettua loppuun.

Ilmaa syötetään järjestelmään pääasiallisen palamisreaktion edistämiseksi (pääasiallinen ilma).
Toisaalta sen lisäksi järjestelmään täytyy syöttää lisäilmaa paikkaamaan epätäydellistä
polttoaineksen ja hapen seosta. Tämän `ylimääräinen` ilma laskee järjestelmän
kokonaistehoa.
