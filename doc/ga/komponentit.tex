\subsection{Komponentit} \label{komponentit}

\subsubsection{Populaatio}

Geneettisessä algoritmissa ratkaisujen joukkoa kutsutaan populaatioksi. Populaation jäseniä
manipuloimalla ja yhdistelemällä luodaan niistä uusia sukupolvia, jotka tavoittelevat edelleen
tulevissa sukupolvissa parempia ratkaisujoukkoja kunnes geneettiselle algoritmille asetettu ehto
on täyttynyt. Populaation koko on syytä valita huolella; liian iso populaatio hidastaa algoritmin
etenemistä ja liian pienellä populaatiolla ei välttämättä saada tarpeeksi hajontaa seuraavan
populaation ratkaisujen muodostamiseen.

\subsubsection{Genotyyppi}
Kussakin genotyypissä esillään populaation jäsen tietokoneen ymmärtämässä muodossa. Genotyypit voidaan
edelleen kääntää \textit{fenotyypeiksi}, eli yksilöksi tai ratkaisuksi.

\subsubsection{Alleeli}
Alleeli käsittää tiedon yhden genotyypin tekijän paikasta ja arvosta.
Alleelin arvot voidaan valita sovelluskohteesta riippuen binääri-, kokonais- tai liukulukuvuiksi.
