\subsection{Algoritmin eteneminen}

\begin{enumerate}
	\item Luodaan populaatio, jonka alkioiden arvot asetetaan tässä työssä alussa satunnaisiksi
	\item Populaatiosta luodaan uusi sukupolvi \textbf{(Tämä kohta ohitetaan ensimmäisellä iteraatiolla)}
	\begin{enumerate}
		\item Alkioihin voidaan tehdä satunnaisia mutaatioita
		\item Tämän jälkeen niiden välillä saatetaan tehdään ristetys, josta syntyy uusi alkio.
		Alkio voi myös siirtyä muuttumattomana seuraavaan sukupolveen, jos esimerkiksi
		populaatiossa tehtävää risteytysten määrää on rajattu.
	\end{enumerate}
	\item Genotyyppi~\(\rightarrow\)~fenotyyppi
	\item	Kunkin fenotyypin perusteella lasketaan alkiolle kelvollisuusarvo
	\item Fenotyyppi~\(\rightarrow\)~genotyyppi
	\item Kelvollisuusarvon perusteella määritellään genotyypin todennäköisyys
	siirtää geenejään seuraavan sukupolveen.
	\item Palataan kohtaan 2, ellei keskeytykselle asetettu ehto toteudu.
\end{enumerate}
