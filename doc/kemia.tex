\section{Kemiallinen tarkastelu}

Puuaines koostuu pääosin selluloosasta, hemiselluloosasta ja ligniinistä
(eli siis hiilestä, typestä, vedystä ja hapesta). Tässä työssä ei
oteta muita aineita huomioon niiden vähäisen vaikuttavuuden ja esiintyvyyden takia.
Puun puhtaan palamisen lopputuotteita ovat vesi ja hiilidioksidi.
Jotta saavutettaisiin mahdollisimman puhdas palamisprosessi, tulee olla sopivasti
aikaa, tarpeeksi korkea lämpötila ja oikeanlainen ilmavirtaus, joka sekoittaa hapen
poltettaviin kaasuihin.

Ideaalisen palamisen
\(a C_A H_B O_C + bO_2 + cN_2 \Rightarrow dCO_2 + bN_2 + eH_2O\) lopputuotteita on
typpi, hiilidioksidi ja vesi\ldots Tosiasiassa osa aineksesta jää palamatta

\begin{equation}
	a C_A H_B O_C + bO_2 + cN_2 \Rightarrow dCO_2 + bN_2 + eH_2O + fCO + gO_2 + hC_x H_y
\end{equation}

\subsection{Vaiheet}

Puun palaminen voidaan jakaa taulukossa~\ref{table:vaiheet} esitettyihin vaiheisiin.
\textbf{Kuivumisen} aikana puusta haihtuu sen sisältämä kosteus. Kosteuden haihtuminen
vaatii energiaa ja se hidastaa palamisprosessia. \textbf{Pyrolyysissä} puuaineksen
molekyylit hajoavat ja muodostavat palavia kaasuja. Pyrolyysin jälkeen alkaa
\textbf{Haihtuvien aineiden palaminen}. Palamista hidastaa samassa tilassa oleva
hiilimonoksi sekä puun kuivumisessa vapautuva vesihöyry, ja sen ylläpito vaatii
riittävää ilmavirtaa. Lopulta tapahtuu \textbf{hiilijäännöksen palaminen},
joka vaatii hapen ja puun pinnalla olevan hiilen yhdistymistä. Lopullinen palaminen
tapahtuu vasta, kun puun pinnalta ei enää haihta kuivumisen tai pyrolyysin seurauksena
vapautuvia kaasuja.

\begin{table}[H]
	\caption{Puun palamisen vaiheet}
	\label{table:vaiheet}
	\begin{tabular*}{\textwidth}{lll}
		\toprule
		\bf Vaihe & \bf Lämpötila (\textdegree C) & \bf Vaatimukset \\
		\midrule
		Kuivuminen & --200 & Energia lämpenemiseen \\
		Pyrolyysi & 200-- & Energia lämpenemiseen \\
		Haihtuvien aineiden palaminen & 225-- & Ilmaa \\
		Hiilijäännöksen palaminen & 225-- & Ilmaa \\
		\bottomrule
	\end{tabular*}
\end{table}

Taulukossa~\ref{table:vaiheet} esitetyt lämpötilat ovat suuntaa-antavia ja
riippuvaisia käytettävän puun koostumuksesta.

Varsinaisen palamisprosessin voidaan kuitenkin katsoa tapahtuvan kahdessa vaiheessa.
Ensimmäisen vaiheen aikana hiili muodostaa ympäristön hapen kanssa hiilimonoksidia \textit CO (häkää).
Toisen vaiheen aikana happi yhdistyy hiilimonoksidin kanssa ja muuttuu hiilidioksidiksi \textit{\(CO_2\)}.

\subsection{Selluloosan palamisreaktio}

\begin{equation}
	C_{6}H_{10}O_{5} + 6O_{2} \Rightarrow 6CO_{2} + 5H_{2}O
\end{equation}

\subsection{Ligniinin palamisreaktio}

\begin{equation}
	\Rightarrow
\end{equation}
