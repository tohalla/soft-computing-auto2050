\documentclass[12pt]{article}
	\usepackage[margin=0.7in]{geometry}
	\usepackage[subpreambles=true]{standalone}
	\usepackage[finnish]{babel}
	\usepackage{
		booktabs,
		bookmark,
		array,
		hyperref,
		graphicx,
		import,
		float,
		amsmath,
		amssymb,
		mathtools,
		textcomp,
		amsmath,
		chngcntr
	}
	\usepackage[utf8]{inputenc}

\restylefloat{table}

\renewcommand*{\tableautorefname}{Taulukko}
\counterwithin*{equation}{section}
\renewcommand{\theequation}{\arabic{section}.\arabic{equation}}

\hypersetup{colorlinks, citecolor=black, filecolor=black, linkcolor=black, urlcolor=black}

\title{\normalsize\uppercase{auto2050}\\\Huge Optimointi geneettistä algoritmia käyttäen}
\author{Touko Hallasmaa}
\date{}

\selectlanguage{finnish}

\begin{document}

\maketitle
\tableofcontents

\begin{abstract}
Geneettinen algoritmi (GA) on luonnolliseen valinnan inspiroima optimointitekniikka.
Sen aikana käytetään \textit{periytymisen}, \textit{mutaatioiden} ja \textit{rekombinaatioiden}
prosesseja.
\end{abstract}

\section{Geneettinen algoritmi}

\subsection{Kelvollisuusfunktio}


\section{Algoritmin eteneminen}

\begin{enumerate}
	\item Luodaan populaatio, jonka alkioiden arvot asetetaan tässä työssä alussa satunnaisiksi
	\item Populaatiosta luodaan uusi sukupolvi \textbf{(Tämä kohta ohitetaan ensimmäisellä iteraatiolla)}
	\begin{enumerate}
		\item Alkioihin voidaan tehdä satunnaisia mutaatioita
		\item Tämän jälkeen niiden välillä saatetaan tehdään ristetys, josta syntyy uusi alkio.
		Alkio voi myös siirtyä muuttumattomana seuraavaan sukupolveen, jos esimerkiksi
		populaatiossa tehtävää risteytysten määrää on rajattu.
	\end{enumerate}
	\item Genotyyppi~\(\rightarrow\)~fenotyyppi
	\item	Kunkin fenotyypin perusteella lasketaan sille kelvollisuusarvo
	\item Fenotyyppi~\(\rightarrow\)~genotyyppi
	\item Kelvollisuusarvon perusteella määritellään genotyypin todennäköisyys
	selviytyä seuraavaan sukupolveen, ja osa populaatiosta eliminoidaan.
	\item Palataan kohtaan 2, ellei keskeytykselle asetettu ehto toteudu.
\end{enumerate}

\section{Työn suunnittelu}

\section{Toteutus}

\subsection{Tehokkuuden mittaus}
Tehokkuutta mitataan tuotetun lämpöenergia ja käytettävän polttoaineen välisenä
suhteena. Suurin mahdollinen arvo on 1, jolloin tapahtuu täydellinen palaminen.

\begin{itemize}
	\item Lämmönsiirron tehokkuus
		\begin{itemize}
			\item Kuinka hyvin polttoastia siirtää energiaa ympäristöön.
			\item astian massa, kyky siirtää tai säilyttää lämpöä, lämpöhäviöt (hormi)
		\end{itemize}
	\item lämpöhäviöitä voi mitata järjestelmästä poistuvien pakokaasujen, jotka
	sisältävät energiaa polttamattoman kaasun ja partikkelimaterian muodossa, perusteella.
\end{itemize}

\subsection{Työn ulkopuolelle rajatut tekijät}

\subsection{Tekijöitä, joita optimoijan käyttäjä pystyy määrittelemään}
\begin{tabular*}{\textwidth}{lcl}
	\toprule
	\bf Nimi & \bf Suure & \bf Kuvaus \\
	\midrule
	Paloaika & \(t\) & Kuinka pitkän aikaa palamista tarkastellaan? \\
	Massa & \(m\) & Poltettavan puun massa \\
	\bottomrule
\end{tabular*}

\subsection{Geneettinen algoritmi}
Alkion genotyypin alleelit ovat liukulukuja väliltä \([0, 1]\).

\subsubsection{Mutaatio}
\subsubsection{Risteytys}


\begin{thebibliography}{9}

	\bibitem{wiki_ga}
	\url{https://en.wikipedia.org/wiki/Genetic_algorithm}

\end{thebibliography}

\end{document}
