\documentclass[12pt]{article}
	\usepackage[margin=0.7in]{geometry}
	\usepackage[subpreambles=true]{standalone}
	\usepackage[finnish]{babel}
	\usepackage{
		booktabs,
		bookmark,
		array,
		hyperref,
		graphicx,
		import,
		float,
		amsmath,
		amssymb,
		mathtools,
		textcomp,
		amsmath,
		chngcntr
	}
	\usepackage[utf8]{inputenc}

\restylefloat{table}

\renewcommand*{\tableautorefname}{Taulukko}
\counterwithin*{equation}{section}
\renewcommand{\theequation}{\arabic{section}.\arabic{equation}}

\hypersetup{colorlinks, citecolor=black, filecolor=black, linkcolor=black, urlcolor=black}

\title{\normalsize\uppercase{auto2050}\\\Huge Optimointi geneettistä algoritmia käyttäen}
\author{Touko Hallasmaa}
\date{}

\selectlanguage{finnish}

\begin{document}

\maketitle
\tableofcontents

\begin{abstract}
Geneettinen algoritmi (GA) on luonnolliseen valinnan inspiroima optimointitekniikka.
Sen aikana käytetään \textit{periytymisen}, \textit{mutaatioiden} ja \textit{rekombinaatioiden}
prosesseja.
\end{abstract}

\section{Geneettinen algoritmi}

Geneettinen algoritmi (GA) on luonnolliseen valinnan inspiroima optimointitekniikka.
Sen aikana käytetään \textit{periytymisen}, \textit{mutaatioiden} ja \textit{rekombinaatioiden}
prosesseja.

\begin{enumerate}
	\item Luodaan populaatio, jonka alkioiden arvot asetetaan tässä työssä alussa satunnaisiksi
	\item Populaatiosta luodaan uusi sukupolvi \textbf{(Tämä kohta ohitetaan ensimmäisellä iteraatiolla)}
	\begin{enumerate}
		\item Alkioihin voidaan tehdä satunnaisia mutaatioita
		\item Tämän jälkeen niiden välillä saatetaan tehdään ristetys, josta syntyy uusi alkio.
		Alkio voi myös siirtyä muuttumattomana seuraavaan sukupolveen, jos esimerkiksi
		populaatiossa tehtävää risteytysten määrää on rajattu.
	\end{enumerate}
	\item Genotyyppi~\(\rightarrow\)~fenotyyppi
	\item	Kunkin fenotyypin perusteella lasketaan sille kelvollisuusarvo
	\item Kelvollisuusarvon perusteella lasketaan genotyypin todennäköisyys selviytyä seuraavaan sukupolveen
	\item Eliminoidaan alkioita edellisessä kohdassa laskettujen todennäköisyyksien avulla
	\item Fenotyyppi~\(\rightarrow\)~genotyyppi
	\item Palataan kohtaan 2, ellei keskeytykselle asetettu ehto toteudu.
\end{enumerate}

\input{algoritmi}
\section{Työn suunnittelu} \label{suunnittelu}

Työssä luodaan sovellus, jolla on mahdollista laskea ohjelmakoodissa määriteltyjä ongelmia geneettisen algoritmin avulla.
Ongelmaa varten tulee määritellä funktio kelvollisuusarvon laskemista varten, sekä siinä esiintyvät muuttujat.
Lisäksi voidaan määrätä geneettinen algoritmi käyttämään jotain ennalta luotua valintaprosessia, jolla valitaan kunkin syklin aikana
populaation jäsenet, jotka siirtävät informaatiota seuraavaan populaatioon.

Ennen sovelluksen kehittämistä valittiin yhtälöt \ref{eq:xyz_graph} ja \ref{eq:sum}, jolla geneettisen algoritmin
aikana luotujen populaatioiden sopivuutta olisi helppo tarkastella. Yhtälössä \ref{eq:xyz_graph} voidaan visualisoida
populaatioiden kehitystä graafisesti kuvaajalla \ref{fig:xyz_graph}. Yhtälön \ref{eq:sum} teoreettinen suurin ja pienin
voidaan helposti johtaa, kun on määritelty siinä käytettävien muuttujien suurimmat ja pienimmät mahdolliset arvot. Esitetyistä yhtälöistä
luodaan maksimointiongelmat, jotka voidaan suorittaa ohjelmasta.

\begin{equation}
	\label{eq:xyz_graph}
	f(x,y) = y \sin{ \left( \sqrt {x^2 + y^2} \right) } + x \left( sign(y) \right)
\end{equation}

\begin{equation}
	\label{eq:sum}
	f(\vec a) = a_1 - a_2 + a_3 - a_4 + a_5 - a_6 + a_7 - a_8 \qquad
	\{\vec a \in \mathbb R^8\}
\end{equation}

\begin{figure}[H]
	\caption{
		Yhtälössä \ref{eq:xyz_graph} esitetyn funktion kuvaaja \(xyz\)-tasossa välillä
		\((x,y,z) = ([-20, 20], [-20, 20], [-40, 40])\)
	}
	\centering
	\includegraphics[width=12cm]{xyz_graph}
	\label{fig:xyz_graph}
\end{figure}

Koska tarkasteltaville muuttujille annetaan arvo niille määritetyn maksimi- ja minimiarvon väliltä, päädyttiin
genotyyppien alleelit esittämään liukulukuarvoin. Alleelit voivat saada lukuarvoja väliltä \([0, 1]\), ja kun
genotyypit käännetään fenotyypeiksi, skaalataan kunkin muuttujaan liitetyn alleelin arvon
perusteella luku muuttujan maksimi ja minimiarvon välillä. Kun fenotyypin arvot on johdettu, voidaan ne
edelleen syöttää funktioon, joka laskee alkiolle kelvollisuusarvon.

Mutaation todennäköisyys on syytä pitää matalana, sillä muutoin geneettinen algoritmi muuttuu liian satunnaiseksi.
Jos alleelien arvot olisivat muodoltaan binäärilukuja, mutaatio voitaisiin toteuttaa siten, että mutatoituvissa
alleeleissa arvo 0 muutetaisiin arvoon 1. Koska työssä olevien alleelien arvot ovat liukulukuja väliltä \([0,1]\),
toteutetaan mutaatio siten, että mutatoitavaan alleeliin lisätään satunnainen arvo joka voi olla myös negatiivinen.
Jotta mutaatiosta saadaan hienovaraisempaa, satunnainen arvo lasketaan siten,
että mitä lähempänä nollaa arvo on, sen todennäköisempää sen valinta on.
Työssä kaikille genotyypin alleeleille annetaan mahdollisuus mutaatioon,
joten yhden syklin aikana alkioon voi tapahtua teoriassa yhtä monta mutaatiota, kun siinä on alleeleja.

Sovellus toteutetaan oliopohjaisena Scala -ohjelmointikieltä käyttäen. Kutakin ongelmaa varten luodaan valvoja (\textit{Overseer})
luokan olio, joka vastaa geneettisen algoritmin suorittamisesta ja sen parametrien hallinnasta.
Ohjelmaan ei tule graafista käyttöliittyymää ja sitä ohjataan komentoriviltä.
Ohjelman käyttäjän tulee voida valita ratkaistava ongelma, populaation koko,
mutaation ja risteytyksen todennäköisyys, sekä pystyä hallinnoimaan ongelmaa varten määriteltyjen muuttujien arvoja.

\section{Toteutus}

\subsection{Tehokkuuden mittaus}
Tehokkuutta mitataan tuotetun lämpöenergia ja käytettävän polttoaineen välisenä
suhteena. Suurin mahdollinen arvo on 1, jolloin tapahtuu täydellinen palaminen.

\begin{itemize}
	\item Lämmönsiirron tehokkuus
		\begin{itemize}
			\item Kuinka hyvin polttoastia siirtää energiaa ympäristöön.
			\item astian massa, kyky siirtää tai säilyttää lämpöä, lämpöhäviöt (hormi)
		\end{itemize}
	\item lämpöhäviöitä voi mitata järjestelmästä poistuvien pakokaasujen, jotka
	sisältävät energiaa polttamattoman kaasun ja partikkelimaterian muodossa, perusteella.
\end{itemize}

\subsection{Työn ulkopuolelle rajatut tekijät}

\subsection{Tekijöitä, joita optimoijan käyttäjä pystyy määrittelemään}
\begin{tabular*}{\textwidth}{lcl}
	\toprule
	\bf Nimi & \bf Suure & \bf Kuvaus \\
	\midrule
	Paloaika & \(t\) & Kuinka pitkän aikaa palamista tarkastellaan? \\
	Massa & \(m\) & Poltettavan puun massa \\
	\bottomrule
\end{tabular*}

\subsection{Geneettinen algoritmi}
Alkion genotyypin alleelit ovat liukulukuja väliltä \([0, 1]\).

\subsubsection{Mutaatio}
\subsubsection{Risteytys}


\begin{thebibliography}{9}

	\bibitem{wiki_ga}
	\url{https://en.wikipedia.org/wiki/Genetic_algorithm}

\end{thebibliography}

\end{document}
