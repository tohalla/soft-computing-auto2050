\section{Geneettinen algoritmi}

Geneettinen algoritmi on luonnollisen valinnan inspiroima stokastinen,
eli satunnaisuuteen perustuva, etsimis- ja optimointiprosessi.
Sen aikana käytetään \textit{periytymisen}, \textit{mutaatioiden} ja \textit{risteytysten}
prosesseja, joilla pyritään löytämään riittävän soveltuva ratkaisu tarkasteltavaan
ongelmaan.

Geneettinen algoritmi eroaa muista tavanomaisemmista haku ja optimointimenetelmistä
siten, että sen aikana käsitellään pikemminkin parametreille asetettuja arvoja, kuin parametreja itsessään.
Näin ollen ratkaisut tehdään parametrien avulla lasketun tuloksen perusteella sen sijaan, että käytettäisiin apuna
jotain johdettua informaatiota, kuten derivaattoja. Tavanomaisesti parametrien avulla lasketun arvon
perusteella määritellään kyseisen arvojoukon \textit{kelvollisuus}, ja siten mahdollisuus siirtää tietoa seuraavaan
optimointisykliin.

GA kannattaa käyttää ongelmissa, joissa ei välttämättä tarvita parasta mahdollista ratkaisua,
vaan riittävän hyvän tuloksen saavuttaminen katsotaan sopivaksi. Menetelmä sopii hyvin myös
ongelmiin, joissa erillaisten ratkaisujen arvojoukko on laaja ja monimutkainen. Näin ollen menetelmää
voidaan käyttää esimerkiksi NP-täydellisten ongelmien lähes optimaalisten ratkaisuje etsimiseen.
Geneettisen algoritmin tukena on lisäksi mahdollista käyttää muita menetelmiä.
Geneettinen algoritmi on hyvin skaalautuva, ja sitä voidaan käyttää useista
muuttujista riippuvien ratkaisujen etsimiseen.
GA:n eduksi voidaan katsoa myös se, että menetelmässä löydetään useita hyviä ratkaisuja.

Mutaatioilla on iso rooli GA:n toiminnassa, sillä ne ovat ainoa tapa tuoda uutta informaatiota populaatioon.
Mutaation aikana joitain \textit{genotyypin}, eli tietokonemuotoisen ratkaisun, arvoja muutetaan
sattumanvaraisesti. Risteytyksessä luodaan uusia alkioita yhdistelemällä kahden tai useamman populaation
jäsenen sisältämää informaatiota eri menetelmin.

\subsection{Komponentit} \label{komponentit}

\subsubsection{Populaatio}

Geneettisessä algoritmissa ratkaisujen joukkoa kutsutaan populaatioksi. Populaation jäseniä
manipuloimalla ja yhdistelemällä luodaan niistä uusia sukupolvia, jotka tavoittelevat edelleen
tulevissa sukupolvissa parempia ratkaisujoukkoja kunnes geneettiselle algoritmille asetettu ehto
on täyttynyt. Populaation koko on syytä valita huolella; liian iso populaatio hidastaa algoritmin
etenemistä ja liian pienellä populaatiolla ei välttämättä saada tarpeeksi hajontaa seuraavan
populaation ratkaisujen muodostamiseen.

\subsubsection{Genotyyppi}
Kussakin genotyypissä esillään populaation jäsen tietokoneen ymmärtämässä muodossa. Genotyypit voidaan
edelleen kääntää \textit{fenotyypeiksi}, eli yksilöksi tai ratkaisuksi.

\subsubsection{Alleeli}
Alleeli käsittää tiedon yhden genotyypin tekijän paikasta ja arvosta.
Alleelin arvot voidaan valita sovelluskohteesta riippuen binääri-, kokonais- tai liukulukuvuiksi.

\input{ga/algoritmi}
