\section{Johdanto}

Työssä tutustutaan Geneettiseen Algoritmin (GA) toimintaan ja sovelletaan sitä Scala -ohjelmointiympäristössä.
Lopputuloksena syntyy ohjelma, jota voidaan käyttää yhtälöiden, sekä maksimointi ja minimointi ongelmien ratkaisuun.

Työssä annetaan joitain esimerkkejä geneettisen algoritmin käytöstä ja tarkastellaan annettujen esimerkkitapausten
implementoimista ja eri tekniikoiden vaikutusta geneettisen algoritmin etenemisessä.
Luotua työkalua ei sellaisenaan ole mahdollista käyttää kaikkiin mahdollisiin
geneettisen algoritmin sovelluksiin.
